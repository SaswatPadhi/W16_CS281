\documentclass[usletter]{article}
\usepackage[margin=1.5in]{geometry}

\usepackage{graphicx}
\usepackage{amsfonts}
\usepackage{amsthm}
\usepackage{amsmath}
\usepackage{amssymb}
\usepackage[mathscr]{eucal}
\usepackage{cancel}
\usepackage{scribe}

\newcommand {\namedset}[1]     {\ensuremath{\mathbb{#1}}}
\newcommand {\langset}[1]      {\ensuremath{\mathcal{#1}}}
\newcommand {\machine}[1]      {\ensuremath{\mathscr{#1}}}
\newcommand {\langfunc}        {\ensuremath{\mathfrak{L}}}
\newcommand {\namedlangset}[1] {\ensuremath{\textnormal{\textsc{#1}}}}
\newcommand {\family}[1]       {\ensuremath{\textnormal{\textsf{#1}}}}

\newcommand {\term}[1]      {\textit{#1}}
\newcommand {\namethm}[1]   {\term{#1} Theorem}
\newcommand {\usingthm}[1]  {(Using \namethm{#1})}

\newcommand {\indpar}[1]   {
  \par\leftskip=#1em
  \noindent\ignorespaces
}
\newenvironment{turing}[2] {
  \smallskip
  \indpar{2}
  \textbf{Machine:} #1\\
  \textbf{Input:} $#2$\\[5pt]
  \textbf{begin}
  \parskip=0pt
  \indpar{3}
}{
  \indpar{2}
  \textbf{end}
  \par\medskip
}

\renewcommand {\theenumi}{\textbf{\arabic{enumi}}}
\renewcommand {\labelenumi}{\theenumi}
\setlength \labelsep {\dimexpr\labelsep + 1em\relax}

\newcommand{\ie}{\textnormal{i.e.}}
\newcommand{\todo}[1]{{\large \textbf{TODO: #1}}}

%%% Defaults for languages and machines
\newcommand {\langL}          {\langset{L}}
\newcommand {\machineM}       {\machine{M}}
\newcommand {\allstrings}     {\ensuremath{\{0,1\}^*}}
\newcommand {\polystrings}[1] {\ensuremath{\{0,1\}^{|#1|^c}}}
%%%

\begin{document}

\makeheader {Saswat Padhi} {February 17, 2016}
            {CS 281A -- Computability and Complexity}

\begin{enumerate}
  \item Given languages $\langL_1$, $\langL_2$, $\langL_3 \in \family{NP}$,
        prove that $\langL_1 \cup (\langL_2 \cap \langL_3) \in \family{NP}$.
  \begin{proof}
    By definition of \family{NP}, the following $3$ Turing machines exist:
    \begin{align}
      \label{q1_m1} \exists \textnormal{ a non-deterministic poly-time TM }
                      \machineM_1 \textnormal{ which decides } \langL_1 \\
      \label{q1_m2} \exists \textnormal{ a non-deterministic poly-time TM }
                      \machineM_2 \textnormal{ which decides } \langL_2 \\
      \label{q1_m3} \exists \textnormal{ a non-deterministic poly-time TM }
                      \machineM_3 \textnormal{ which decides } \langL_3
    \end{align}
    Then the following TM $\machineM'$ exists, using these machines:
    \begin{turing}{$\machineM'$}{x}
      non-deterministically simulate $\machineM_1(x)$,
        and return $1$ if it returns $1$ \\
      non-deterministically simulate $\machineM_2(x)$,
        and return $0$ if it returns $0$ \\
      non-deterministically simulate $\machineM_3(x)$,
        and return $0$ if it returns $0$ \\
      return $1$
    \end{turing}
    Clearly, $\machineM'$ is a (non-deterministic) poly-time TM;
    because simulating (three) poly-time TMs only incurs \textit{logarithmic}
    time overhead, as in case of a \term{Universal TM}. \\
    Concretely, the running time of $\machineM'$ would be
    $O((k_1 |x|^{k_1} + k_2 |x|^{k_2} + k_3 |x|^{k_3}) \log(|x|))$
    where $\machineM_i$ runs in $O(|x|^{k_i})$;
    or equivalently $O(|x|^{\max(k_1, k_2, k_3)})$.
    Thus, $\langfunc(\machineM') \in \family{NP}$.

    \begin{claim}
      $\forall x : x \in \langL_1 \cup (\langL_2 \cap \langL_3)
                  \Rightarrow x \in \langfunc(\machineM') \ie
        \langL_1 \cup (\langL_2 \cap \langL_3) \subseteq \langfunc(\machineM')$.
    \end{claim}
    \begin{proof}
      Assume an arbitrary string input $x \in \allstrings$.
      \begin{align*}
        x \in \langL_1 \cup (\langL_2 \cap \langL_3)
          &\Rightarrow x \in \langL_1
              \vee (x \in \langL_2 \wedge x \in \langL_3) \\
          &\Rightarrow \machineM_1(x) = 1
            \vee (\machineM_2(x) = 1 \wedge \machineM_3(x) = 1)
            & \textnormal{(Using \ref{q1_m1}, \ref{q1_m2}, \ref{q1_m3})} \\
          &\Rightarrow \machineM'(x) = 1 & \textnormal{(by construction)}
      \end{align*}
      Therefore,
      $\langL_1 \cup (\langL_2 \cap \langL_3) \subseteq \langfunc(\machineM')$.
    \end{proof}

    \begin{claim}
      $\langfunc(\machineM') \subseteq \langL_1 \cup (\langL_2 \cap \langL_3)$
      \ie $\forall x : x \in \langfunc(\machineM')
                      \Rightarrow x \in \langL_1 \cup (\langL_2 \cap \langL_3)$.
    \end{claim}
    \begin{proof}
      Assume an arbitrary string input $x \in \allstrings$.
      By construction of $\machineM'$:
      \begin{align*}
        &x \in \langfunc(\machineM') \ \ie \ \machineM'(x) = 1 \\
        &\Rightarrow \machineM_1(x) = 1
          \vee \neg ((\machineM_1(x) \neq 1 \wedge \machineM_2(x) = 0)
                     \vee (\machineM_1(x) \neq 1 \wedge \machineM_2(x) \neq 0
                                                 \wedge \machineM_3(x) = 0)) \\
        &\textnormal{(simplification)} \\
        &\Rightarrow \machineM_1(x) = 1
          \vee (\machineM_2(x) = 1 \wedge \machineM_3(x) = 1) \\
        &\Rightarrow x \in \langL_1
          \vee (x \in \langL_2 \wedge x \in \langL_3) \\
        &\Rightarrow x \in \langL_1 \cup (\langL_2 \cap \langL_3)
      \end{align*}
      Therefore,
      $\langfunc(\machineM') \subseteq \langL_1 \cup (\langL_2 \cap \langL_3)$.
    \end{proof}

    Thus, we have proved that
    $\langfunc(\machineM') =
     \langL_1 \cup (\langL_2 \cap \langL_3) \in \family{NP}$.
  \end{proof}

\begin{remark}
    In fact, we can always treat unions as disjunctions and intersections as
    conjunctions over machine outputs;
    to prove that \family{NP} is closed under these operations.
  \end{remark}

  \item Prove that $\lfloor n^{\frac{2016}{2015}} \rfloor$
        is \term{time-constructible}.
  \begin{proof}
    We need to show that, there exists some TM \machineM,
    which maps $1^n \mapsto 1^{\lfloor n^{2016/2015} \rfloor}$
    in time $O(\lfloor n^{\frac{2016}{2015}} \rfloor)$.
    \todo{solve}
  \end{proof}

  \item Consider the following task: given the encoding $\alpha \in \allstrings$
        of a TM $\machineM_\alpha$ that halts on \textit{every} input,
        determine whether $\machineM_\alpha$ runs in constant time.
        Either give an algorithm for this problem
        or prove that it is undecidable.
  \begin{proof}[Solution]
    \todo{solve}
  \end{proof}

  \item Let \langL\ denote the set of langauges decidable by TMs with a
        \namedlangset{Halt} oracle.
        Prove that their exists a language not in \langL.
        Can you define it explicitly?
  \begin{proof}
    \begin{figure}[h]
    \centering
    \def\arraystretch{1.8}\tabcolsep=6pt
    \begin{tabular}{|c||c|c|c|c|c|c|c|c|c|c|c|c|}
      \hline
      \textbf{\langL} \textbackslash~$x$
        & 0 & 1 & 00 & 01 & 10 & 11 & \ldots & $x_i$ & $x_{i+1}$ & \ldots
        & $x_j$ & \ldots \\\hline\hline

      $\langL_{0}$ \hfill & \cancel{0}~\textbf{1} & 1 & 0 & 1 & 0 & 0 & \ldots
                          & 1 & 0 & \ldots & 1 & \ldots \\\hline
      $\langL_{1}$ & 0 & \cancel{0}~\textbf{1} & 1 & 0 & 1 & 1 & \ldots & 1 & 1
                   & \ldots & 1 & \ldots \\\hline
      $\langL_{00}$ \hfill & 1 & 0 & \cancel{1}~\textbf{0} & 1 & 0 & 1 & \ldots
                           & 0 & 0 & \ldots & 0 & \ldots \\\hline
      $\langL_{01}$ \hfill & 1 & 0 & 0 & \cancel{1}~\textbf{0} & 0 & 0 & \ldots
                           & 1 & 1 & \ldots & 0 & \ldots \\\hline
      \ldots & \ldots & \ldots & \ldots & \ldots & \ldots & \ldots & \ldots
             & \ldots & \ldots & \ldots & \ldots & \ldots \\\hline
      $\langL_{i}$ \hfill & 1 & 0 & 0 & 1 & 1 & 0 & \ldots
                          & \cancel{1}~\textbf{0} & 1 & \ldots & 0
                          & \ldots \\\hline
      $\langL_{i+1}$ & 0 & 1 & 1 & 1 & 0 & 0 & \ldots & 1
                     & \cancel{0}~\textbf{1} & \ldots & 0 & \ldots \\\hline
      \ldots & \ldots & \ldots & \ldots & \ldots & \ldots & \ldots & \ldots
             & \ldots & \ldots & \ldots & \ldots & \ldots \\\hline
      $\langL_{j}$ \hfill & 0 & 0 & 0 & 1 & 1 & 0 & \ldots & 1 & 1 & \ldots
                          & \cancel{0}~\textbf{1} & \ldots \\\hline
      \ldots & \ldots & \ldots & \ldots & \ldots & \ldots & \ldots & \ldots
             & \ldots & \ldots & \ldots & \ldots & \ldots \\\hline
    \end{tabular}
    \caption{We enumerate the languages $\langL_m \in \langL$, decidable by the
      oracle machines equipped with \namedlangset{Halt}, and all strings in
      $\{0, 1\}^*$. Note that the enumeration of $\langL_m$ might be arbitrary.
      In particular $m$ need not have any relation to the oracle machine whose
      language is $\langL_m$. In the above table, $\left< \langL_m, x_n \right>$
      cell is $1$ if $x_n \in \langL_m$, else $0$.  Using a typical
      diagonalization argument, we can construct the language
      $\langL_{Diag} = \{ x_n \mid x_n \not\in \langL_n \}$
      s.t. $\langL_{Diag} \not\in \langL$.}
    \label{diag_table}
    \end{figure}

    We would prove this using diagonalization. Consider all languages decidable
    using oracle machines equipped with \namedlangset{Halt}, and all strings in
    $\{0,1\}^*$. As shown in Figure \ref{diag_table}; by simply inverting the
    memberships of diagonal elements i.e the cells $\left< \langL_n,x_n \right>$
    , we have a new language $\langL_{Diag}$ which we show that, does not belong
    to \langL.
    \begin{equation}
      \label{L_Diag}
      \langL_{Diag} = \{ x_n \mid x_n \not\in \langL_n \}
    \end{equation}
    \begin{claim}
      $ \langL_{Diag} \not\in \langL $.
    \end{claim}
    \begin{proof}
      For the sake of contradiction, assume $\langL_{Diag} \in \langL$. Then,
      there must exist some language $\langL_m \in \langL$ in Figure
      \ref{diag_table} s.t. $\langL_m = \langL_{Diag}$.

      Now consider the string $x_m$ and the cell $\left< \langL_m, x_m \right>$.
      There are only 2 possible values for the cell: $0$ or $1$, or equivalently
      either $x_m \in \langL_m$ or $x_m \not\in \langL_m$.

      \textsc{Case 1: $x_m \in \langL_m$}
      \begin{align*}
        x_m \in \langL_m \quad &\Rightarrow \quad x_m \not\in \langL_{Diag}
            & (\textnormal{definition from Equation }\ref{L_Diag})\\
          &\Rightarrow \quad x_m \not\in \langL_m
            & (\textnormal{assumed }\langL_m = \langL_{Diag})
      \end{align*}
      We thus have a contradiction in case 1!

      \textsc{Case 2: $x_m \not\in \langL_m$}
      \begin{align*}
        x_m \not\in \langL_m \quad &\Rightarrow \quad x_m \in \langL_{Diag}
            & (\textnormal{definition from Equation }\ref{L_Diag})\\
          &\Rightarrow \quad x_m \in \langL_m
            & (\textnormal{assumed }\langL_m = \langL_{Diag})
      \end{align*}
      Again, we have a contradiction in case 2 as well!

      Thus, our assumption  $\langL_{Diag} \in \langL$ must be false, and hence
      $\langL_{Diag} \not\in \langL$.
    \end{proof}

    Note that, since we did not assume any order of enumeration for $\langL_m$
    in Figure \ref{diag_table}, the membership check $x_n \in \langL_n$ required
    for $\langL_{Diag}$ is effectively checking membership of an arbitrary
    string in an arbitrary language. Given the fact that these languages are
    decidable by oracle machines equipped with \namedlangset{Halt}, we can
    re-write $\langL_{Diag}$ as:
    $$
      \langL_{Diag} = \{ \left< \alpha , x \right> \mid
                         \machineM_\alpha^\namedlangset{Halt}(x) = 0 \}
    $$
    which is the (co-)\term{Halting Problem} defined on oracle machines equipped with
    \namedlangset{Halt}!
  \end{proof}

  \item Prove that if $\family{PH} = \family{PSPACE}$,
        then the polynomial hierarchy collapses.
  \begin{proof}
    From lecture \cite{lec6}, we know that
    $\namedlangset{TQBF} \in \family{PSPACE-C}$ (\family{PSPACE}-complete).
    So, we have $\namedlangset{TQBF} \in \family{PH-C}$ by our assumption.
    As discussed in class, just the \textit{non-emptiness} of \family{PH-C},
    collapses the polynomial hierarchy!
    \todo{write}
  \end{proof}

  \item Prove that
        $\family{P}^\namedlangset{TQBF} = \family{NP}^\namedlangset{TQBF}$.
  \begin{proof}
    This proof would be very similar to a proof we did in class, which showed
    $\exists \textnormal{ an oracle } \langset{O}:
      \family{P}^\langset{O} = \family{NP}^\langset{O}$.
    We showed that $\family{EXP} \subseteq \family{P}^\langset{O}
                                 \subseteq \family{NP}^\langset{O}
                                 \subseteq \family{EXP}$.
    Similarly, we would prove in this case, that
    $\family{PSPACE} \subseteq \family{P}^\namedlangset{TQBF}
                     \subseteq \family{NP}^\namedlangset{TQBF}
                     \subseteq \family{PSPACE}$.

    It trivially holds that
    $\family{P}^\namedlangset{TQBF} \subseteq \family{NP}^\namedlangset{TQBF}$;
    since every deterministic poly-time TM is also a non-deterministic poly-time
    TM. Thus, we only have to prove the following:

    \begin{claim}
      $ \family{PSPACE} \subseteq \family{P}^\namedlangset{TQBF} $.
    \end{claim}
    \begin{proof}
      We know that $\namedlangset{TQBF} \in \family{PSPACE-C}$. So we have,
      $$\namedlangset{TQBF} \preceq_p \langL \quad \ie \quad
      x \in \langL \Longleftrightarrow p_\langL^\namedlangset{TQBF} (x)
        \in \namedlangset{TQBF} $$
      where $\preceq_p$ denotes poly-time reduction, and
      $p_\langL^\namedlangset{TQBF} (x)$ is a poly-time transformation function,
      which can transform an instance of \langL to that of \namedlangset{TQBF}.

      Then following TM equipped with oracle \namedlangset{TQBF} would be able
      to decide \langL:
      \begin{turing}{$\machineM_\langL^\namedlangset{TQBF}$}{x}
        apply the polynomial time transformation
          $p_\langL^\namedlangset{TQBF}(x)$ resulting in $y$ \\
        \texttt{query} oracle \namedlangset{TQBF} for $y$ \\
        return the result from the oracle
      \end{turing}
      Clearly,
      $\forall \langL \in \family{PSPACE} :
        \langL = \langfunc(\machineM_\langL^\namedlangset{TQBF})
        \wedge \langfunc(\machineM_\langL^\namedlangset{TQBF})
          \in \family{P}^\namedlangset{TQBF}$.

      Thus, $\family{PSPACE} \subseteq \family{P}^\namedlangset{TQBF}$.
    \end{proof}

    \begin{claim}
      $ \family{NP}^\namedlangset{TQBF} \subseteq \family{NPSPACE} $.
    \end{claim}
    \begin{proof}
      By definition a TM in \family{NP}, uses at most polynomial amount of space
      and thus $\family{NP} \subseteq \family{NPSPACE}$. \\
      We also know that $\namedlangset{TQBF} \in \family{PSPACE}$. So we can
      encode the TM for \namedlangset{TQBF} into our non-deterministic TM,
      instead of using it as an oracle, and we would still use polynomial amount
      of space (if we re-use the same cells for computations every time we
      invoke this TM). \\
      Thus, instead of relativizing if we simply embed the TM for \family{TQBF},
      we have a non-deterministic TM with the same language which still in
      polynomial space.
    \end{proof}

    Therefore, we have
    $\family{PSPACE} \subseteq \family{P}^\namedlangset{TQBF}
                     \subseteq \family{NP}^\namedlangset{TQBF}
                     \subseteq \family{NPSPACE}
                     \subseteq \family{PSPACE}$
    (the last containment is due to Savitch's Theorem);
    from which it follows immediately that
    $\family{P}^\namedlangset{TQBF} = \family{NP}^\namedlangset{TQBF}$.
  \end{proof}

  \item Prove that $\family{NTIME}(n^{2016}) \subsetneq \family{PSPACE}$.
  \begin{proof}
    \todo{solve}
  \end{proof}

  \item Is there an algorithm that evaluates totally quantified Boolean formulas
        of size $n$ in space $O(n^{1/2016})$? Prove your answer.
  \begin{proof}[Solution]
    \todo{solve}
  \end{proof}
\end{enumerate}

\bibliographystyle{abbrv}
\bibliography{midterm}

\end{document}
