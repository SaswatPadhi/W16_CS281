\documentclass[usletter]{article}
\usepackage[margin=1.5in]{geometry}

\usepackage{amsfonts}
\usepackage{amsmath}
\usepackage{amssymb}
\usepackage{amsthm}
\usepackage{cancel}
\usepackage[mathscr]{eucal}

\usepackage{enumitem}
\usepackage{graphicx}

\usepackage{scribe}

\newcommand {\namedset}[1]     {\ensuremath{\mathbb{#1}}}
\newcommand {\langset}[1]      {\ensuremath{\mathcal{#1}}}
\newcommand {\machine}[1]      {\ensuremath{\mathscr{#1}}}
\newcommand {\langfunc}        {\ensuremath{\mathfrak{L}}}
\newcommand {\namedlangset}[1] {\ensuremath{\textnormal{\textsc{#1}}}}
\newcommand {\family}[1]       {\ensuremath{\mathsf{#1}}}

\newcommand {\term}[1]      {\textit{#1}}
\newcommand {\namethm}[1]   {\term{#1} Theorem}
\newcommand {\usingthm}[1]  {(Using \namethm{#1})}

\newcommand {\reduce}[2]    {\ensuremath{\preceq_{#1}^{#2}}}
\newcommand {\complete}[2]  {\ensuremath{\reduce{#1}{#2}     %% ugly hack ahead!
                                         \hspace*{-4pt}\textnormal{-complete}}}

\newcommand {\indpar}[1]   {
  \par\leftskip=#1em
  \noindent\ignorespaces
}
\newenvironment{turing}[2] {
  \smallskip
  \indpar{2}
  \textit{Machine:} #1\\
  \textit{Input:} $#2$\\[5pt]
  \texttt{begin}
  \parskip=0pt
  \indpar{3}
}{
  \indpar{2}
  \texttt{end}
  \par\medskip
}

\newcommand{\ie}{\textnormal{i.e. }}
\newcommand{\todo}[1]{{\large \textbf{TODO: #1}}}

%%% Defaults for languages and machines
\newcommand {\langL}          {\langset{L}}
\newcommand {\machineM}       {\machine{M}}
%%%

%%% Complexity Classes
\newcommand {\PTime}  {\family{P}}
\newcommand {\PH}     {\family{PH}}
\newcommand {\IP}     {\family{IP}}
\newcommand {\NP}     {\family{NP}}
\newcommand {\RP}     {\family{RP}}
\newcommand {\BPP}    {\family{BPP}}
\newcommand {\PSPACE} {\family{PSPACE}}
\newcommand {\FPi}    {\family{\Pi}}
\newcommand {\FSigma} {\family{\Sigma}}
%%%

\begin{document}

\makeheader {Saswat Padhi} {March 18, 2016}
            {CS 281A -- Computability and Complexity}

\begin{enumerate}[labelsep=2.5em, label=\textbf{\arabic{enumi}}]
  %%%%%%%%%%%%%%%%%%%%%%%%%%%%%%%%%%%%%%%%%%%%%%%%%%%%%%%%%%%%%%%%%%%%%%%%%%%%%%
  % Q1
  \item A language $\langL$ is \term{Turing-recognizable} if there is a Turing
        machine that halts if and only if the input is in \langL. Prove for an
        explicit language that it is not Turing-recognizable.
  %-----------------------------------------------------------------------------
  \begin{proof}[Intuition]
    We know that problem \namedlangset{Halt} is \term{undecidable}. So, at least
    one of \namedlangset{Halt}, or $\overline{\namedlangset{Halt}}$ is not
    Turing-recognizable.
    But \namedlangset{Halt} is. So, $\overline{\namedlangset{Halt}}$ must not be!
  \end{proof}
  \begin{proof}
    Consider the complement language of \namedlangset{Halt}:
    $$ \overline{\namedlangset{Halt}}
    = \{ \langle \alpha, x \rangle \mid
         \machineM_\alpha \text{ never halts on } x \} $$
    We would prove that $\overline{\namedlangset{Halt}}$ is not
    Turing-recognizable.

    We begin by demonstrating that \namedlangset{Halt} is indeed
    Turing-recognizable.
    \begin{claim}\namedlangset{Halt} is Turing-recognizable.\end{claim}
    \begin{proof}
      Consider a universal TM \machine{U}. \\
      By construction, \machine{U} halts on an input $\langle \alpha, x \rangle$
      if and only if $\machineM_\alpha$ halts on $x$; because \machine{U} just
      simulates $\machineM_\alpha$ with at most a logarithmic time overhead.

      Clearly, \machine{U} thus \term{recognizes} \namedlangset{Halt}.
    \end{proof}

    \begin{assumption}
      For the sake of contradiction; assume the existence of a TM \machine{V}
      which recognizes $\overline{\namedlangset{Halt}}$ i.e. \machine{V} halts if
      and only if $\machineM_\alpha$ never halts on $x$.
    \end{assumption}

    Now consider the following TM, which we can construct using \machine{U} and
    \machine{V}:
    \begin{turing}{\machine{H}}{\langle  \alpha, x \rangle}
      simulate next step of \machine{U} on $\langle  \alpha, x \rangle$,
      halt with $1$ if \machine{U} halted. \\
      simulate next step of \machine{V} on $\langle  \alpha, x \rangle$,
      halt with $0$ if \machine{V} halted. \\
      repeat the above again
    \end{turing}

    \begin{claim}
      The TM \machine{H} \term{decides} \namedlangset{Halt}.
    \end{claim}
    \begin{proof}
      It is easy to see that \machine{H} halts for all inputs. Consider any
      input $\langle \alpha, x \rangle$. Then one of the following must hold:
      \begin{itemize}
        \item $\machineM_\alpha$ halts on on $x$, then \machine{U} must
          eventually halt. \\
          Thus \machine{H} must halt with $1$.
        \item $\machineM_\alpha$ never halts on $x$, then (\machine{U} never
          halts and) \machine{V} must eventually halt. \\
          Thus \machine{H} must halt with $0$.
      \end{itemize}
      Therefore, \machine{H} must eventually halt, in any case.

      Further, \machine{H} halts with $1$ on input $\langle \alpha, x \rangle$
      if and only if \machine{U} halts on the same input i.e. $\machineM_\alpha$
      halts on $x$. In other words, \machine{H} decides \namedlangset{Halt}.
    \end{proof}

    Hence, we have shown that using \machine{V}; we can \term{decide}
    \namedlangset{Halt} -- which is \term{undecidable}! \\
    Then our assumption regarding the existence of such a TM \machine{V} which
    recognizes $\overline{\namedlangset{Halt}}$ must be false.

    Therefore, $\overline{\namedlangset{Halt}}$ is not
    \term{Turing-recognizable}.
  \end{proof}


  %%%%%%%%%%%%%%%%%%%%%%%%%%%%%%%%%%%%%%%%%%%%%%%%%%%%%%%%%%%%%%%%%%%%%%%%%%%%%%
  % Q2
  \item Give a polynomial-time algorithm for determining whether a given 2-CNF
        formula is satisfiable.
  %-----------------------------------------------------------------------------


  %%%%%%%%%%%%%%%%%%%%%%%%%%%%%%%%%%%%%%%%%%%%%%%%%%%%%%%%%%%%%%%%%%%%%%%%%%%%%%
  % Q3
  \item Let \langL\ be any \NP-complete language. Prove that $\langL \in \BPP$,
        if and only if $\langL \in \RP$.
  %-----------------------------------------------------------------------------


  %%%%%%%%%%%%%%%%%%%%%%%%%%%%%%%%%%%%%%%%%%%%%%%%%%%%%%%%%%%%%%%%%%%%%%%%%%%%%%
  % Q4
  \item Prove that there is a real number $0 < p < 1$ with the following
        property: given access to a coin with bias $p$, a randomized Turing
        machine can decide an undecidable language in polynomial time. By
        \textit{access} to the coin is meant the ability to receive any desired
        number of random bits $X_1, X_2, X_3 ... \in \{0,1\}$, each being an
        independent random variable with expectation $p$.
  %-----------------------------------------------------------------------------


  %%%%%%%%%%%%%%%%%%%%%%%%%%%%%%%%%%%%%%%%%%%%%%%%%%%%%%%%%%%%%%%%%%%%%%%%%%%%%%
  % Q5
  \item Prove that the polynomial hierarchy collapses if \namedlangset{TQBF} has
        a polynomial-time \textit{randomized} algorithm.
  %-----------------------------------------------------------------------------
  \begin{proof}
    If \namedlangset{TQBF} has a polynomial-time randomized algorithm,
    $\namedlangset{TQBF} \in \BPP$.

    Due to \namethm{Sipser-G\'{a}cs-Lautemann}\cite{Lautemann1983}:
    $\BPP \subseteq \FSigma_2 \cap \FPi_2$.
    Then, $\namedlangset{TQBF} \in \FSigma_2$.

    But we know that \namedlangset{TQBF} is \complete{p}{} for \PSPACE. So, any
    problem $\langset{P} \in \PSPACE$ can be reduced to \namedlangset{TQBF} in
    polynomial-time. Since $\PH \subseteq \PSPACE$, the reduction also holds for
    any problem $\langset{P} \in \PH$.

    But since we have assumed that $\namedlangset{TQBF} \in \FSigma_2$, we have
    actually shown that: \\
    $$ \forall \langset{P} \in \PH: \langset{P} \in \FSigma_2 $$
    In other words, the polynomial hierarchy \PH\ collapses to the $2^\text{nd}$
    level.
  \end{proof}


  %%%%%%%%%%%%%%%%%%%%%%%%%%%%%%%%%%%%%%%%%%%%%%%%%%%%%%%%%%%%%%%%%%%%%%%%%%%%%%
  % Q6
  \item Two players take turns choosing vertices from a directed graph $G$, such
        that the $i^\text{th}$ vertex chosen $v_i$, is distinct from $v_0, v_1,
        v_2, ..., v_{i-1}$ and reachable from $v_{i-1}$ by an edge. When a
        player has no vertices to choose from, the game ends and that player
        loses. On input a directed graph $G$, give a polynomial-space algorithm
        for deciding whether the player who starts the game can force a win,
        regardless of his opponent's choices. \\
        \textit{Comment:} This problem makes it clear why \PSPACE\ is often
        regarded as the class of adversarial games with two players.
  %-----------------------------------------------------------------------------


  %%%%%%%%%%%%%%%%%%%%%%%%%%%%%%%%%%%%%%%%%%%%%%%%%%%%%%%%%%%%%%%%%%%%%%%%%%%%%%
  % Q7
  \item Prove that increasing the completeness parameter in the definition of
        \IP\ from $\frac{2}{3}$ to $1$, does not change this complexity class.
        What class do we get if we keep the completeness unchanged but increase
        the soundness from $\frac{2}{3}$ to $1$?
  %-----------------------------------------------------------------------------


  %%%%%%%%%%%%%%%%%%%%%%%%%%%%%%%%%%%%%%%%%%%%%%%%%%%%%%%%%%%%%%%%%%%%%%%%%%%%%%
  % Q8
  \item Prove that with Hadamard coding, it is in general impossible to recover
        the original codeword, if the received transmission is corrupted in
        $25\%$ or more of the coordinates.
  %-----------------------------------------------------------------------------

\end{enumerate}

\bibliographystyle{abbrv}
\bibliography{final}

\end{document}
